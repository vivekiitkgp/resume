\documentclass{article}
\usepackage[cm]{fullpage}
\usepackage{color}
\usepackage{hyperref}
\usepackage{multicol}
\usepackage{microtype}
\DisableLigatures{encoding = *, family = *}


\setlength{\columnsep}{1cm}
\hypersetup{breaklinks=true,%
colorlinks=true,%
linkcolor=cyan,%
urlcolor=MyDarkBlue}

\definecolor{MyDarkBlue}{rgb}{0,0.0,0.45}

%%%%%%%%%%%%%%%%%%%%%%%%%%
% Formatting parameters  %
%%%%%%%%%%%%%%%%%%%%%%%%%%

\newlength{\tabin}
\setlength{\tabin}{0.5em}
\newlength{\secsep}
\setlength{\secsep}{0.175cm}

\setlength{\parindent}{0in}
\setlength{\parskip}{0in}
\setlength{\itemsep}{0in}
\setlength{\topsep}{0in}
\setlength{\tabcolsep}{0in}

\definecolor{contactgray}{gray}{0.3}
\pagestyle{empty}

%%%%%%%%%%%%%%%%%%%%%%%%%%
%  Template Definitions  %
%%%%%%%%%%%%%%%%%%%%%%%%%%

\newcommand{\lineunder}{\vspace*{-8pt} \\ \hspace*{-6pt} \hrulefill \\ \vspace*{-15pt}}
\newcommand{\name}[1]{\begin{center}\textsc{\Huge#1}\\\end{center}}
\newcommand{\program}[1]{\begin{center}\textsc{#1}\end{center}}
\newcommand{\contact}[1]{\begin{center}\color{contactgray}{\small#1}\end{center}}

\newenvironment{tabbedsection}[1]{
  \begin{list}{}{
      \setlength{\itemsep}{0pt}
      \setlength{\labelsep}{0pt}
      \setlength{\labelwidth}{0pt}
      \setlength{\leftmargin}{\tabin}
      \setlength{\rightmargin}{\tabin}
      \setlength{\listparindent}{0pt}
      \setlength{\parsep}{0pt}
      \setlength{\parskip}{0pt}
      \setlength{\partopsep}{0pt}
      \setlength{\topsep}{#1}
    }
  \item[]
}{\end{list}}

\newenvironment{nospacetabbing}{
    \begin{tabbing}
}{\end{tabbing}\vspace{-1.2em}}

\newenvironment{resume_header}{}{\vspace{0pt}}


\newenvironment{resume_section}[1]{
  \filbreak
  \vspace{2\secsep}
  \textsc{\large#1}
  \lineunder
  \begin{tabbedsection}{\secsep}
}{\end{tabbedsection}}

\newenvironment{resume_subsection}[2][]{
  \textbf{#2} \hfill {\footnotesize #1} \hspace*{-3.5em}
  \begin{tabbedsection}{0.5\secsep}
}{\end{tabbedsection}}

\newenvironment{subitems}{
  \renewcommand{\labelitemi}{$\cdot$}
  \begin{itemize}
      \setlength{\labelsep}{1em}
}{\end{itemize}}

\newenvironment{resume_employer}[4]{
  \vspace{\secsep}
  \textbf{#1} \\ 
  \indent {\small #2} \hfill\hspace{1em}{\footnotesize#3 (#4)}
  \begin{tabbedsection}{0pt}
  \begin{subitems}
}{\end{subitems}\end{tabbedsection}}


%%%%%%%%%%%%%%%%%%%%%%%%%%
%     Start Document     %
%%%%%%%%%%%%%%%%%%%%%%%%%%

\begin{document}

\begin{resume_header}
\name{Vivek Rai}
\program{Undergraduate student at IIT Kharagpur \\ A303, LBS Hall of Residence, India}
\contact{vivekrai@iitkgp.ac.in/+91-8013291569 \hspace{1cm} \url{https://vivekiitkgp.github.io}}
\end{resume_header}

\begin{resume_section}{Interests}
Computational Biology, Bioinformatics, Machine Learning, Sequence Analysis, and Systems Biology.
\end{resume_section}


\begin{resume_section}{Education}
  \begin{resume_subsection}[Kharagpur, WB (2012-2017 expected)]{Indian Institute of Technology Kharagpur}
    \begin{subitems}
      \item Bachelor's and Master's degree in Biotechnology and Biochemical engineering with \textbf{GPA 8.53 out of 10},
      \item Pursuing Minor in Mathematics and Computing,
      \item \textbf{Ranked 2} in class of 50 students,
      \item Completed 2 additional non-departmental courses with \textbf{8.5 GPA}.
    \end{subitems}
  \end{resume_subsection}

  \begin{resume_subsection}[Kolkata, WB (Till 2012)]{Shree Jain Vidyalaya}
    \begin{subitems}
      \item Cumulative average of 93\% \& 80\% in final high school and senior high school examinations respectively,
      \item \textbf{Awarded}: Best Student Award, Scholarship for 5 years of schooling during 2007-2012, etc.,
    \end{subitems}
  \end{resume_subsection}
\end{resume_section}


\begin{resume_section}  {Work Experience}
  \begin{resume_subsection}[\url{http://github.com/yannickwurm/sequenceserver}]{SequenceServer \\ \footnotesize {Dr. Yannick Wurm}}
    \begin{subitems}
      \item The project aims to provide biologists with an intuitive and easy to setup custom BLAST server to effectively query and handle large sequence data, \textbf{paper \emph{in prep}};
      \item Implemented \textbf{BLAST+} output parser module and back-end data-layer in \textbf{Ruby}, thereby improving application architecture, usability, and modularity;
      \item Designed graphical overview scheme for obtained hit information using \textbf{d3.js} ({\footnotesize \url{http://www.d3js.org}}), a Javascript visualization library, total contribution translates to over \textbf{100} commits and 8 months of activity. \\
    \end{subitems}
    \end{resume_subsection}

  \begin{resume_subsection}[Apr, 2014]{Sign Language Interpreter \\ \footnotesize {Prof. P. Patnaik}}
  \begin{subitems}
    \item Conceived and designed a gesture to text (or speech) application to interpret sign language gestures (non-motion) with a team of 4 people for aiding deaf and dumb people;
    \item Implemented image processing techniques to obtain noise free information from real time video; classified data into relevant clusters and predicted unknown information using \textbf{k-means clustering};
    \item Exploring further possibility of providing service through chat applications or online widget/web based services.\\
    \end{subitems}
  \end{resume_subsection}

  \begin{resume_subsection}[Mar, 2014]{Jigsaw Puzzle Solver \\ \footnotesize {Prof. S.K. Barai}}
    \begin{subitems}
    \item Evaluated different techniques based on \textbf{Genetic Algorithm} to solve large piece jigsaw puzzle (randomly shuffled pieces of an image); implemented mutation strategies; came up with an approach to use this technique to solve images with non unique components;
    \item Programmed the algorithm entirely from scratch in C++ using OpenCV image processing libraries; could solve up to 1000 pieces.\\
    \end{subitems}
  \end{resume_subsection}

  \begin{resume_subsection}[Mar, 2014]{Automated Torn Paper Mosaicing}
  \begin{subitems}
    \item Collaborated with team to develop and implement algorithms to digitally stitch manually torn paper pieces to reconstruct original one with minimal loss;
    \item Familiarized myself with Object Oriented Design pattern, \textbf{OpenCV} image processing algorithms (Canny, Douglas-Peucker etc.,), feature extraction and analysis techniques.
    \end{subitems}
  \end{resume_subsection}
  
\end{resume_section}

\begin{resume_section}{Coursework \footnote{Online courses not mentioned}}

  \begin{resume_subsection}[\footnotesize {Supervisor: Prof. S.K. Barai}]{Term Paper \\ Comparison of Fuzzy Guided Gene Prediction Methods}
    \begin{subitems}
      \item Reviewed different state-of-art techniques to analyze and annotate whole organism's genome in an automated way to predict genes and other regions of interest;
      \item Critiqued the future prospects and application strategies of \textbf{SVM, NN} learning and heuristic techniques (\textbf{GA, Fuzzy Logic}) as hybrid methods for better annotation of raw genomic data.\\
    \end{subitems}
    \end{resume_subsection}
    
\begin{resume_subsection}[(T)heory and (L)aboratory classes]{Core Courses}
\vspace*{-8pt}
    \begin{subitems}
        \begin{multicols}{2}
        \item Cell and Molecular Biology (T/L)
        \item Microbiology (T/L)
        \item Genetics
        \item Biochemistry (T/L)
        \item Biochemical, and Bio analytical Labs.
        \item Bioinformatics (T/L)$^{\#}$
        \item Protein Engineering$^{\#}$
        \item Probability and Statistics
        \item Statistical Decision Modelling
        \item Mathematics I \& II
        \end{multicols}
    \end{subitems}
\end{resume_subsection}

\begin{resume_subsection}[\#To be completed by Spring 2015]{Additional Courses}
\vspace*{-8pt}
    \begin{subitems}
        \begin{multicols}{2}
        \item Discrete Structures
        \item Soft Computing Tools in Engineering
        \end{multicols}
    \end{subitems}
\end{resume_subsection}

\end{resume_section}

\begin{resume_section}{Skills}
    \begin{resume_subsection}[]{Laboratory Experience:}
    \vspace*{-8pt}
    \begin{subitems}
        \begin{multicols}{3}
        \item Microscopy
        \item Aseptic Techniques
        \item Centrifugation
        \item Staining, Culture, and \\ Isolation of Microorganisms
        \item Cell Fractionation
        \item Assay techniques
        \item HPLC, FPLC
        \item Spectrophotometry \& \\ Spectrofluorometry
        \item Gas/Column Chromatography
        \item Gel Electrophoresis
        \item DNA Amplification (PCR)
        \item DNA, RNA \& Protein Isolation \\ and Purification
        \end{multicols}
    \end{subitems}
    \end{resume_subsection}
\begin{resume_subsection}{Programming Skills:}
  \begin{nospacetabbing}
  \textbf{Production Quality  {\hskip 2em}} \= Python (scipy stack), JavaScript, Ruby\\*
  \textbf{Dabbled In  } \> Haskell, C, R, BASH, Node.js, d3.js, \LaTeX \\*
  \textbf{Platforms  } \> Unix (primary), Windows\\*
  \textbf{Bioinformatics  } \> BLAST+, PyMol, Sequence Analysis, BioPython\\*
  \textbf{Practices and Tools} \> Git, Scientific Computing, Design Patterns\\*
  \end{nospacetabbing}
\end{resume_subsection}
\end{resume_section}

\begin{resume_section}{Extra Curricular Activities}
    \begin{subitems}
    \item Conceptualized and promoted campaigns to increase participation of students from village communities for further schooling on voluntary basis,
      \item Co-organized multiple hackathons, online coding competitions, a Google blogger challenge, and other activities as a member of official Google Students Club,
      \item Submitted about 10 articles for \emph{Alankar}, college's annual magazine for graduating students,
      \item Led a team of 6 people for participation in Inter Hall \emph{Opensoft} competition, an annual software design competition,
      \item Authored over \textbf{30 articles} and more than \textbf{1600 edits} to the English Wikipedia,\footnote{http://en.wikipedia.org/wiki/User:Vivek\_Rai}.
    \end{subitems}
\end{resume_section}


\end{document}
